\section{Introduction}
This roadmap for the Precursor VC Fund describes how TruCol got started, where we currently are, which milestones take us to a significant market share, and what capital is being raised now, and down the road.

\subsection{backstory}
While developing our own software company besides our studies, we wanted to increase the rate of development by setting out bounties. After evaluating bounty platforms, we noticed most take a cut as middle-person, allow for ambiguity, or cater to niche markets. Since we wanted to give the developer the full reward, we assembled a student team from Radboud University in Nijmegen, and Delft University, both in the Netherlands. We combined Aerospace Engineering students, with Artificial Intelligence students and Computer Science students and competed at the ETHDenver to develop the protocol.

The TruCol protocol presents an improvement of market efficiency and developer autonomy by decentralisation and automation of test-driven development. The protocol promotes inclusive, fair and accessible work, by enabling developers to participate in the market regardless of their circumstances. Employers publish a smart contract with a bounty for deterministically verifiable development tasks which are fit for solving by external parties. Developers from all over the world are able to complete these tasks and get rewarded automatically when the requirement of the smart contract is fulfilled. The protocol thus removes the middleman and costly fees, and stimulates an open and fair development market.

This work was awarded multiple prizes, amongst others, a prize of \$3000,- for contributing to UN sustainability goal nr 8:"Providing fair and equal work to all". We continued the development and generated a POC in Solidity. This POC is presented during the Ethereum Conference 4 in Paris in the summer of 2021.
