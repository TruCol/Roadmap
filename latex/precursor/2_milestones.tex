\section{Milestones}
Six major milestones are identified within the TruCol project. To get to an operational break-even position, the first two are relevant. The latter two are relevant for future seeding rounds and exponential growth.

\subsection{Operational Break Even}
\begin{enumerate}
    \item \textit{Complete CI deployment} - To improve development collaboration, our self-hosted GitLab-CI should run in a Docker container or Virtual Machine, instead of on our own devices. This is to prevent interference, and to help filter code contributions that do not adhere to a minimum quality. Given the adversarial nature of this protocol, that code quality is essential.
    \item  \textit{Support all languages} - To allow any company to use the protocol, we should extend the protocol from a Solidity-Solidity implementation only, to oracles that query the (decentralised) CI build status of the bounty hunter solutions.
    \item \textit{First Customer Usage} - We have a first customer, Viggo Service Enablers, an airport logistics company, that is eager to use the protocol once available.
\end{enumerate}
% TODO: include milestones in Gantt Chart

\subsection{Growth}
\begin{enumerate}
    \item \textit{Wide-spread Adoption} - Over 100.000 bounties must have been deployed, with a net value of at least \$30.000.000 in bounties being allocated. This provides us with a minimal amount of data and revenue potential to start working on an in-house arbitrage AI engine.
    \item  \textit{Exit/Next Seeding Round} - To build an arbitrage AI, we need to raise funds. It is expected this requires North of 100 million, given the complexity of the task.
    \item \textit{Self-sustainable AI} - If we succeed to build a self-sustaining arbitrage AI, we can improve software development efficiencies worldwide, the software development landscape has changed into requirement specifications.
\end{enumerate}

\noindent The remainder of this roadmap focusses on the roadmap to operational break even.
