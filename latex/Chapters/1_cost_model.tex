\section{Cost Model}\label{sec:cost_model}
\noindent This section describes the expected costs to bring the TruCol company to an operational break-even. These costs are described in the form of a cost model, that is described in \cref{subsec:model_description} and the model parameters are given in \cref{subsec:model_parameters}. The mathematical model specification and assumptions are located in the appendices at \cref{sec:cost_model_specification} and \cref{sec:assumptions} respectively.

\subsection{Description}\label{subsec:model_description}
\noindent The total costs are computed based on the following 4 factors:
\begin{itemize}
	\item The cumulative amount of human labour hours that are planned to be executed, multiplied with their respective hourly wage costs.
	\item Bounty subsidisation to attract new users.
	\item Buffer costs to generate wide-spread adoption of the TruCol protocol.
    \item Daily operational costs, e.g. electricity, company phone usage, travel etc.
\end{itemize}
\noindent For the mathematical specification of this model, see \cref{sec:cost_model_specification}. For the Python code of this model, one can check: \url{https://github.com/TruCol/Roadmap/tree/main/src}.

\subsection{Model Parameters}\label{subsec:model_parameters}

\ifx\homepath\overleafhome
    % Overleaf compilation.
    \begin{longtable}{@{}cp{.7\textwidth}@{}}
    \caption{Cost Model Parameters in \euro\label{table:nonlin}}\\
    \toprule
    {\bfseries Parameter} & {\bfseries Value} \\ \midrule
    \endfirsthead
    \caption{Cost Model Parameters in \euro (continued)}\\
    \toprule
    \multicolumn{2}{l}{\scriptsize\emph{\ldots{} continued}}\\
    {\bfseries Parameter} & {\bfseries Value} \\ \midrule
    \endhead
    \multicolumn{2}{r}{\scriptsize\emph{to be continued\ldots}}\\
    \bottomrule
    \endfoot
    \bottomrule
    \endlastfoot
    blockchain dev & 76\\
    front end dev & 41\\
    human resources & 36\\
    bounty subsidising & 100000\\
    buffer & 100000\\
\end{longtable}
 %\newpage
\else
    % Local compilation
    \begin{longtable}{@{}cp{.7\textwidth}@{}}
    \caption{Cost Model Parameters in \euro\label{table:nonlin}}\\
    \toprule
    {\bfseries Parameter} & {\bfseries Value} \\ \midrule
    \endfirsthead
    \caption{Cost Model Parameters in \euro (continued)}\\
    \toprule
    \multicolumn{2}{l}{\scriptsize\emph{\ldots{} continued}}\\
    {\bfseries Parameter} & {\bfseries Value} \\ \midrule
    \endhead
    \multicolumn{2}{r}{\scriptsize\emph{to be continued\ldots}}\\
    \bottomrule
    \endfoot
    \bottomrule
    \endlastfoot
    blockchain dev & 76\\
    front end dev & 41\\
    human resources & 36\\
    bounty subsidising & 100000\\
    buffer & 100000\\
\end{longtable}
 %\newpage
\fi
\noindent For a motivation for these values, see \cref{sec:assumptions}
