\section{Model Description}\label{sec:model_description}
% What to convey (Source: Pitch Perfect):
% What is our Market?
% What is the Market Size?
% What is the Market Trajectory
This


\subsection{Market}\label{subsec:model_description_market}
To compute the TAM, SAM and SOM, some form of market definition can be used. To this end it can be valuable to specify exactly what the TruCol consultancy does, where it adds value and how it does that. Furthermore, since these three afore mentioned estimates pertain to a potential future, the potential, yet deemed feasible, activities of the TruCol consultancy are included.

The TruCol consultancy provides advice and support to companies on how companies can get the most out of the TruCol protocol. To understand this the following assumptions are shared:

\begin{itemize}
	\item \textbf{asu-0:} Task completions of tasks that are completed using the TruCol protocol are deterministically verifyable.
	\item \textbf{asu-1:} Solutions of tasks that are completed using the TruCol protocol are of sufficient quality.
	\item \textbf{asu-2:} Tasks that are completed using the TruCol protocol can be solved for the lowest costprice that is currently available in this world.
	\item \textbf{asu-3:} No personel needs to be attracted, screened, hired nor fired for tasks that are completed using the TruCol protocol.
	\item \textbf{asu-4:} Companies can benefit from public particular solutions to their task specifications. 
	\item \textbf{asu-5:} By sampling from a bigger talent pool (this world), the average performance of the solutions will be better than what is produced by the in-house talent pool, or, for equal solution performance, a faster rate of development can be obtained on average for an equal or lower price.
\end{itemize}

Under these assumptions, one can conclude that an economically rational company would try to off-load as much of their required tasks into the TruCol protocol as it would minimise their operational costs and/or improve algorithmic efficiency of their solutions. 

We help companies identify the tasks for which they can use the TruCol protocol, and we assist them in writing safe test specifications that are not easily hackable. This implies that under the given set of assumptions, the TAM for the TruCol protocol can be defined as the total costs that the companies (and consumers) in this world are willing to pay for assistance on using the TruCol protocol. 

\subsubsection{TruCol Total Addressable Logistics Market}\label{subsubsec:tam_logistics}

This sub-sub section illustrates a rough method of estimating the logistics subsegment of the TAM for the TruCol protocol. To do this, an example of algorithmic optimalisation within the logistics market as presented by McKinsey \& Company is generalised conservatively to a rough estimate of the total logistics market size.

A clear example of a logistics company succesfully hiring a consultancy for algorithmic optimalisation is documented by McKinsey \& Company in the "how they help their clients" segment of their website\cite{mckinsey_algo}. The study how reports McKinsey's team, among which the McKinsey's Strategic Network Analytic Center helped an Asian logistics company. With McKinseys team, the logistics company realised an \textit{in line haul network cost} reduction of 3.6\% while reducing their \textit{transit time} with 0.8\%, yielding an overal 16\% increase in profit for the logistics company, without compromising the quality. To use this report as a valuable resource to generate some rough estimates on market size, the following assumptions are made:

 \begin{itemize}
 	\item \textbf{asu-6:} The logistics company made a net profit by hiring McKinsey \& Company in this particular ordeal.
	\item \textbf{asu-7:} The example of a 16\% increase in profit is generalizable to a conservative potential 0.1\% of profit increases through algorithmic optimalisation accross the entire logistics industry.
	\item \textbf{asu-8:} Companies are willing to pay at least 1 \% of their potential profit increases for the assistance the TruCol consultancy company provides in identifying opportunities for optimisation and for improving test-specification security.
\end{itemize}

Based on those assumptions, one could find a potential yearly profit increase accross the entire logistics sector by summing the net profit of the logistics sector. \cite{cips} claims that this company \cite{transparency_market_research} valued the logistics market at 8.1 trillion in 2016. Additionally \cite{cips} claims \cite{transparency_market_research} estimates the logistics market value will grow to 15.5 trillion in 2023. However, no figures on profit are found. Hence individual companies are explored. 

For DHL one can find on pdf page 37/170 in \cite{dhl_2019_annual_report} that the annual profit for DHL in 2019 was 4.1 billion.

For UPS one can find on pdf page 4/257 in \cite{ups_2020_annual_report} that the annual unadjusted operating profit for UPS in 2020 was 7.7 billion. Note, \cite{ups_q21_earnings_call} says UPS had a net operating profit of 1.1 billion in Q1 of 2020, implying they had to almost double their average profit in the remaining three quarters of 2020 to be consistent with an annual 7.7 billion.

For FedEx the net income as reported for 2020 has been 1.29 \$ billion in pdf page 2/17 \cite{fedex_2020_annual_report}. 

\begin{itemize}
\item \textbf{Asu-9:} The net income as reported (GAAP) by FedEx can be interpreted as the profit by FedEx.
\end{itemize}

Next, the claim that fragmentation of the global market implied in 2016 that Deutsche Post DHL, Ceva Logistics, UPS, and FedEx, control less than 15\% of that global market allows estimating a limit on the net global profit made in the logistics market based on the following assumptions:

\begin{itemize}
	\item \textbf{Asu-10:} The market segment in the global logistics market maintained by the combination of DHL, UPS and FedEx is at most 15\% in 2020.
	\item \textbf{Asu-11:} The profit in the remaining 85\% of the global logistics market has the same average yearly profitability per percent market share as the combination of DHL, UPS and FedEx.
\end{itemize}

Based on assumptions 1-11 one could estimate an upperbound of 
\begin{equation}
	\begin{split}
		net-profit_{DHL+UPS+FedEx}=4.1+7.7+1.29=13.09billion\\
		\frac{net-profit_{global_logistics}}{net-profit_{DHL+UPS+FedEx}}=\frac{0.85}{0.15}\\
		net-profit_{global_logistics}=net-profit_{DHL+UPS+FedEx}\frac{0.85}{0.15}\\
		net-profit_{global_logistics}=\frac{13.09\cdot0.85}{0.15}\\
		net-profit_{global_logistics}=74.2 billion
	\end{split}
\end{equation}

Hence, if each of those companies in the logistics sector could increase their profits on average annually by .1\% using algorithmic optimisation, and if they would use the TruCol protocol to do that, and if they would be willing to invest 1\% of that profit in our support and assistance in getting the most out of the TruCol protocol, we would currently estimate that this would yield roughly an income of $74.2\cdot 0.001\cdot 0.01=\$0.74 million$

\subsubsection{Additional addressable markets}
Since the TruCol consultancy is market agnostic, we also seek to assist in algorithmic optimisation outside the logistics market. Several markets are worth mentioning in particular as we expect them to either heavily rely on algorithmic optimisations, or because they are particularly suited for the TruCol protocol.
\begin{itemize}
	\item \textbf{(Automated) trading} In the highly competitive market of (automated) trading, algorithmic optimisations are key to making successfull trades. 
	\item \textbf{Space Sector} The space engineering sector already has a relatively high test driven development\cite{todo}, this lowers the adoption costs of the TruCol protocol relative to most industries. Furthermore, space applications are heavily mass constrained, which generally makes them highly energy constrained as well. These energy constraints emphasise the importance of algorithmic optimisations, for example in telecomunications satellites and swarm robots.
	\item \textbf{Innovative Materials Research} The domain of material science has been adopting algorithmic search strategies to find new materials  \cite{allahyari2020coevolutionary}.
	\item \textbf{Pharmaceutical Industry} Another example of a large market that has been shifting to adopt algorithmic search strategies to find new medicines.
\end{itemize}
Each of these are multi billion dollar markets which can contribute to the TAM of the TruCol consultancy.
% NP problems
% Neuromorphic
% Space
% Logistics
% Chemical compound development
% Protein folding
\subsection{Emerging markets}
Beyond those listed markets, the following emerging markets could be great opportunities for the TruCol consultancy to latch in and grow along in.
\begin{itemize}
	\item \textbf{Neuromorphic Computing} This field is developing new complexity theory to adapt to the unconventional computation methods. This is an interesting opportunity to explore the versatility of the TruCol protocol.
	\item \textbf{Quantum Computing} This is another upcoming field with many new algorithmic implementations. The newness of the field may suggest that the amount of optimisation and exploration to be done is relatively high, possibly indicating a relatively large potential for the TruCol protocol. However, currently our team does not yet contain experience in this type of algorithmic developments.
	\item \textbf{Artificial Intelligence} With the introduction of GPT 3 the world has seen an example of an AI engine that is able to generate code for some basic tasks \cite{todo}. The TruCol protocol could catalyse the usage of such AI engines based on requirement specification. We expect that users of the TruCol protocol will develop a tactical advantage on requirement specifications for AI engines.
\end{itemize}

% AI requirements specicfication
% AI engines
\subsection{Market Size}\label{subsec:model_description_market_size}
\subsection{Market Trajectory}\label{subsec:model_description_market_trajectory}

Since the market size estimation models are somewhat of an abstract/subjective task, three different approaches are used in an attempt to establish some reference material with respect to accuracy.



Before the model is presented, it is important to realise that we propose a consultancy service that operates as an optimisation service. This means that if a certain activity, e.g. a logistics company has operational cost of 5 \$million/day, our consultancy service is only able to earn at most the margin of improvement we are able to bring our customer. So suppose the independent usage of the TruCol provides the customer with a 2\% optimisation in their operational costs, yielding them $5.000.000\cdot 0.02=100.000/day$\$. Suppose our expertise is able to enable them to yield a 3\% optimisation by identifying the relevant development/system processes and supporting them in improved test specification. In that assumption our consultancy would bring them an additional 3-2=1\% which would translate roughly to $50.000$\$. That would be the value we bring to the logistics company in this hypothetical scenario.

In reality this example is oversimplified, the 2\% the company could get by themselves would involve some risk pertaining to inaccurate test specification which could lead to loss of the bounty. Our company reduces this risk by providing test-specification security expertise. Furthermore, our interaction with the client may bring the client experience that can be applied in future applications of the TruCol protocol, hence the value to we bring to the client is larger than the amount they gain in terms of optimisation w.r.t. the case where they use the protocol themselves.



\subsubsection{Top Down}\label{subsubsec:model_descriptions_top_down}
The Top-Down approach 
\subsubsection{Bottom Up}\label{subsubsec:model_descriptions_bottom_up}
\subsubsection{Value Theory}\label{subsubsec:model_descriptions_value_theory}